\section{Data}

\subsection{Dataset Overview}

We use the \textbf{NFL Big Data Bowl 2023} tracking dataset, which provides
10 Hz player trajectories (positions $x, y$, speed $s$) for all players on the field,
along with play metadata (down, distance, yardline, formation, personnel, defensive team) 
and game context (clock, score differential, quarter). Data spans 8 weeks of regular-season games 
(\texttt{week1.csv}--\texttt{week8.csv}).

\subsection{Size and Modalities}

After filtering to pass plays only (see Section~\ref{sec:filtering}), the dataset contains 
$\sim$8{,}100 plays. Plays are fixed-length sequences of 60 frames (6 seconds at 10 Hz).  
The diffusion model uses all 22 players (11 offense, 11 defense), with fixed player ordering:
offense sorted as QB, RB, WR, TE, OL; defense sorted as DL, LB, DB.

Trajectory features per player: $(x, y, s)$ (3 features), 
yielding a 66-dimensional state vector for 22 players (66D).

Context conditioning includes five categorical variables 
(down, offensive formation, personnel grouping, defensive team, situation category) 
and three continuous variables (yards-to-go, normalized yardline, hash mark position).

\subsection{Splits and Filtering}
\label{sec:filtering}

We use a random 80/20 split of 90\% of all 8 weeks for train/validation, 
with the remaining 10\% held out for testing 
($\sim$5{,}800 train / $\sim$1{,}500 val / $\sim$800 test plays).

Filtering criteria:
\begin{itemize}
    \item \textbf{Pass plays only}: Rush plays (\texttt{passResult == 'R'}) are excluded 
          to focus the model on route generation rather than designed runs.
    \item \textbf{Sequence truncation}: Plays are truncated at \texttt{pass\_outcome\_caught} 
          or \texttt{pass\_outcome\_incomplete} events, capturing only the route phase 
          (before ball is caught or incomplete).
    \item \textbf{Player requirements}: Each play must have exactly 22 tracked players 
          (11 offense, 11 defense).
    \item \textbf{Minimum valid frames}: Plays with fewer than 50\% valid (non-padded) frames 
          are excluded.
\end{itemize}

\subsection{Normalization and Preprocessing}

All plays are standardized so offense always moves left-to-right by flipping coordinates 
when \texttt{playDirection == 'left'} and rotating angles accordingly; 
this removes directional asymmetry while preserving field geometry.

Both autoregressive and diffusion models use \textbf{z-score normalization} per feature:
each coordinate and speed value is standardized to mean~0, standard deviation~1.
Normalization statistics are computed across all plays and stored for denormalization 
during inference.

Sequences are padded or truncated to fixed length $T = 60$ frames with zero-padding 
for shorter plays. Fixed player ordering ensures consistent role assignments across plays.

\subsection{Formation Anchoring}

For the diffusion model, we compute \textbf{formation anchors}---initial player positions 
at $t=0$ based on the offensive formation, personnel grouping, yardline, and hash mark.
These anchors are derived from formation templates (e.g., SHOTGUN, SINGLEBACK, I\_FORM, 
EMPTY, TRIPS, BUNCH, 12\_PERSONNEL) and locked during sampling to ensure realistic 
starting formations. The anchor loss in training penalizes deviations from these 
predefined positions at the initial timestep.

\subsection{Biases and Limitations}

The dataset is imbalanced toward 1st down situations and common formations (shotgun predominates).
Filtering to pass plays removes all rushing attempts, limiting applicability to run-heavy scenarios.
Fixed-length sequences (60 frames) truncate longer plays and pad shorter ones, 
potentially losing late-stage dynamics or introducing artificial static periods.
Formation templates cover common but not all possible formations, 
and the fixed player ordering may not perfectly match all personnel groupings.
Evaluation focuses on offense trajectory quality; 
defensive movement serves as context but is not explicitly evaluated for realism.

